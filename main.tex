\documentclass[aps, prd
, preprint
%, twocolumn
, nofootinbib 
%, notitlepage
%, superscriptaddress
, longbibliography
]{revtex4-1}
\usepackage{graphicx}
\usepackage[caption=false]{subfig}
\usepackage{mathrsfs}
\usepackage{amsmath,amssymb}
\usepackage{bm}
\usepackage{braket}
\usepackage{listings}
\usepackage{cases}
\usepackage{comment}
\usepackage{soul}
\usepackage{cancel}
\usepackage{cases}
\usepackage[utf8]{inputenc}
\usepackage{url}
\usepackage{longtable}
\usepackage[normalem]{ulem}
\usepackage[colorlinks=true
,urlcolor=blue
,anchorcolor=blue
,citecolor=blue
,filecolor=blue
,linkcolor=blue
,menucolor=blue
%,pagecolor=blue
,linktocpage=true
,pdfproducer=medialab
,pdfa=true
]{hyperref}

%\usepackage{mathpazo}
%\usepackage[no-math]{fontspec}
%\setmainfont{Palatino}
%\setsansfont{Optima}

\newcommand{\dif}[2]{\frac{\mathrm{d} #1}{\mathrm{d} #2}}
\newcommand{\pdif}[2]{\frac{\partial #1}{\partial #2}}
\newcommand{\var}[2]{\frac{\delta #1}{\delta #2}}
\newcommand{\dd}{\mathrm{d}}
\newcommand{\DD}{\mathscr{D}}
\newcommand{\ee}{\mathrm{e}}
\newcommand{\diag}{\mathrm{diag}}
\newcommand{\sgn}{\mathrm{sgn}}
\newcommand{\Mpl}{M_\text{Pl}}
\newcommand{\ns}{n_{{}_\mathrm{S}}}
\newcommand{\cs}{c_{{}_\mathrm{S}}}
\newcommand{\IR}{\text{IR}}
\newcommand{\UV}{\text{UV}}
\renewcommand{\Re}{\mathrm{Re}}
\renewcommand{\Im}{\mathrm{Im}}
\newcommand{\dk}{\frac{\dd^3k}{(2\pi)^3}}
\newcommand{\bbalpha}{{\alpha\!\!\!\alpha}}
\newcommand{\dps}{\displaystyle}
\newcommand{\SIA}{S_\text{IA}}
\newcommand{\eff}{\text{eff}}
\newcommand{\kdx}{\mathbf{k}\cdot\mathbf{x}}
\newcommand{\FP}{\text{FP}}

\newcommand{\calD}{\mathcal{D}}
\newcommand{\scrD}{\mathscr{D}}
\newcommand{\calg}{\mathcal{g}}
\newcommand{\calH}{\mathcal{H}}
\newcommand{\scrH}{\mathscr{H}}
\newcommand{\uI}{\text{I}}
\newcommand{\calJ}{\mathcal{J}}
\newcommand{\scrJ}{\mathscr{J}}
\newcommand{\calL}{\mathcal{L}}
\newcommand{\scrL}{\mathscr{L}}
\newcommand{\calN}{\mathcal{N}}
\newcommand{\calO}{\mathcal{O}}
\newcommand{\scrO}{\mathscr{O}}
\newcommand{\calP}{\mathcal{P}}
\newcommand{\calR}{\mathcal{R}}
\newcommand{\uR}{\text{R}}

\newcommand{\bae}[1]{\begin{align} #1 \end{align}}
\newcommand{\bce}[1]{\begin{cases} #1 \end{cases}}
\newcommand{\bfe}[4]{
\begin{figure} 
	\centering
	\includegraphics[#1]{#2}
	\caption{#3}
	\label{#4}
\end{figure}}
\newcommand{\bpme}[1]{\begin{pmatrix} #1 \end{pmatrix}}

\newcommand{\Red}[1]{\textcolor{red}{\sffamily #1}}
\newcommand{\Mag}[1]{\textcolor{magenta}{\sffamily #1}}
\newcommand{\Blue}[1]{\textcolor{blue}{\sffamily #1}}
\newcommand{\mathblue}[1]{\textcolor{blue}{#1}}

\allowdisplaybreaks[4]



\begin{document}
\title{Numerical solver of stochastic-$\delta N$ formalism: \texttt{StocDeltaN}}
\date{\today}

\author{Yuichiro Tada}
\email{tada.yuichiro@e.mbox.nagoya-u.ac.jp}
\affiliation{Department of Physics, Nagoya University, Nagoya 464-8602, Japan}
\affiliation{Institut d'Astrophysique de Paris, GReCO, UMR 7095 du CNRS et Sorbonne Universit\'e, 
98bis boulevard Arago, 75014 Paris, France}


\begin{abstract}
\end{abstract}

\maketitle
%\tableofcontents




\section{Phase space formulation}

\subsection{Basic equations}

Let us consider the following form of the phase space Langevin eq.
\bae{
	\dd\phi^I=h_\phi^I\dd N+g^I_{Qa}\dd W_a, \qquad \dd\pi_I=h_{\pi I}\dd N+g_{PIa}\dd W_a.
}
For a standard choice, the deterministic coefficients are given by
\bae{
	h_\phi^I=G^{IJ}\frac{\pi_J}{H}, \qquad h_{\pi I}=-3\pi_I-\frac{V_I}{H}+\frac{1}{H}\Gamma^K_{IJ}G^{JL}\pi_K\pi_L,
}
with the Hubble parameter calculated by the Friedmann eq.
\bae{
	3\Mpl^2H^2=\frac{1}{2}G^{IJ}\pi_I\pi_J+V.
}
The noise coefficients are determined by the power spectra of linear perturbations. To see the power spectra, it is better to introduce the covariant momenta defined by
\bae{
	\tilde{P}_I=P_I-\Gamma^K_{IJ}\pi_KQ^J.
}
At the leading order in the slow-roll approximation, the correlators between $Q$ and $\tilde{P}$ on the superhorizon scale read
\bae{
	\calP_{QQ}{}^{IJ}\simeq\left(\frac{H}{2\pi}\right)^2G^{IJ}, \qquad \calP_{Q\tilde{P}}{}^I{}_J\simeq\calP_{\tilde{P}\tilde{P}IJ}\simeq0,
}
and therefore those between $Q$ and $P$ are given by
\bae{
	\calP_{QQ}{}^{IJ}\simeq\left(\frac{H}{2\pi}\right)^2G^{IJ}, \qquad \calP_{QP}{}^I{}_J\simeq\Gamma^K_{JL}\pi_K\calP_{QQ}{}^{IL},
	\qquad \calP_{PPIJ}\simeq\Gamma^K_{IL}\Gamma^M_{JN}\pi_K\pi_M\calP_{QQ}{}^{LN}.
}
The noise coefficients are the square-root of these correlators as
\bae{
	g^I_{Qa}\simeq\frac{H}{2\pi}e^I_a, \qquad g_{PIa}\simeq\Gamma^K_{IJ}\pi_Kg^J_{Qa},
}
where $e^I_a$ is a certain choice of the vielbein $e^I_ae^J_a=G^{IJ}$. If one takes account of the mass correction for noise amplitudes,
it would be better to give the vielbein by the adiabatic-entropic decomposition described in Sec.~\ref{sec: Adiabatic-entropic decomposition}.

The corresponding Fokker-Planck (FP) eq. reads
\bae{
	\partial_NP=\calL_{\FP}\cdot P,
}
with the Fokker-Planck operator $\calL_{\FP}$ defined by
\bae{
	\calL_{\FP}\cdot P=-\partial_{\phi^I}(D_{\phi}^IP)-\partial_{\pi_I}(D_{\pi I}P)+\frac{1}{2}\partial_{\phi^I\phi^J}(D_{\phi\phi}{}^{IJ}P)
	+\partial_{\phi^I\pi_J}(D_{\phi\pi}{}^I{}_JP)+\frac{1}{2}\partial_{\pi_I\pi_J}(D_{\pi\pi IJ}P).
}
The coefficients are
\bae{
	\text{It\^o:}& \qquad
	\bce{
		\dps
		D_\phi^I=h_\phi^I, \\
		\dps
		D_{\pi I}=h_{\pi I},
	} \\
	\text{Stratonovich:}& \qquad
	\bce{
		\dps
		D_\phi^I=h_\phi^I+\frac{1}{2}(g^J_{Qa}\partial_{\phi^J}+g_{PJa}\partial_{\pi J})g^I_{Qa}, \\
		D_{\pi I}=h_{\pi I}+\frac{1}{2}(g^J_{Qa}\partial_{\phi^J}+g_{PJa}\partial_{\pi J})g_{PIa}, 
	}
}
and
\bae{
	D_{\phi\phi}{}^{IJ}=g^I_{Qa}g^J_{Qa}=\calP_{QQ}{}^{IJ}, \qquad D_{\phi\pi}{}^I{}_J=g^I_{Qa}g_{PJa}=\calP_{QP}{}^I{}_J, \qquad 
	D_{\pi\pi IJ}=g_{PIa}g_{PJa}=\calP_{PPIJ},
}
both for It\^o and Stratonovich integrals.

For this FP eq., the Vincent's partial differential equation (PDE) is given by
\bae{
	\bce{
		\dps
		\calL_\FP^\dagger\cdot f_n=-nf_{n-1}, \\ 
		\dps
		\calL_\FP^\dagger\cdot g_2=-D_{\phi\phi}{}^{IJ}(\partial_{\phi^I}f_1)(\partial_{\phi^J}f_1)
		-2D_{\phi\pi}{}^I{}_J(\partial_{\phi^I}f_1)(\partial_{\pi_J}f_1)-D_{\pi\pi IJ}(\partial_{\pi_I}f_1)(\partial_{\pi_J}f_1),
	}
}
with $f_0=1$ and the absorbing boundary condition $f_n|_\text{boundary}=g_2|_\text{boundary}=0,\,(n\ge1)$. 
Here $f_n(\phi,\pi)=\braket{\calN^n}(\phi,\pi)$, $g_2(\phi,\pi)=\braket{\delta\calN^2}(\phi,\pi)$,
and $\calL^\dagger_\FP$ denotes the adjoint FP operator 
\bae{
	\calL^\dagger_\FP=D_\phi^I\partial_{\phi^I}+D_{\pi I}\partial_{\pi_I}+\frac{1}{2}D_{\phi\phi}{}^{IJ}\partial_{\phi^I}\partial_{\phi^J}+D_{\phi\pi}{}^I{}_J\partial_{\phi^I}\partial_{\pi_J}
	+\frac{1}{2}D_{\pi\pi IJ}\partial_{\pi_I}\partial_{\pi_J}.
}



\subsection{Numerical implementation}

\subsubsection{JacobiPDE}

\bfe{width=\hsize}{figs/lattice.pdf}{}{fig: lattice}

In this subsubsection, we describe the numerical algorithm to solve PDE implemented in \texttt{JacobiPDE.hpp}.
To solve PDE numerically, one divides the integration region $\Omega$ into the lattice as schematically indicated in Fig.~\ref{fig: lattice}.
In \texttt{JacobiPDE.hpp}, the Jacobi method is used, where the site value is updated by the nearest eight sites for e.g. 2-dim. case and its own self.

First let us specify the boundary condition. To make use of the $\delta N$ formalism, the surface of the end of inflation should be given by the constant energy slice.
Therefore, with a certain threshold energy $\rho_c$, the boundary condition is set as
\bae{
	f_{n(\ge1)}(\phi,\pi)=g_2(\phi,\pi)=0, \qquad \text{where $\rho(\phi,\pi)=\rho_c$ (indicated by the blue line in Fig.~\ref{fig: lattice})}.
}
In the discretized lattice case, the site value where $\rho\le\rho_c$ (the blue-shaded region in Fig.~\ref{fig: lattice}) is fixed to zero and not updated during the calculation.

For a hilltop-type inflation model, the integration region $\Omega$ can be surrounded by this end surface $\rho_c$ and the above boundary condition is sufficient.
However, for e.g. a chaotic-type model, the slow-roll inflationary region in general has boundaries between eternal inflationary regions and then the integration region 
$\Omega$ cannot be completely surrounded by absorbing boundaries like as an example Fig.~\ref{fig: lattice}.
In that case, we impose the reflecting condition on these not-EoI (end of inflation) boundaries. 
The Jacobi method requires the nearest sites for each point, so we introduce the imaginary sites one-step outside of $\Omega$.
The field values of these points are determined by those of the reflecting positions which are inside $\Omega$.
The schematic image is indicated in Fig.~\ref{fig: lattice}. Practically, these not-EoI boundaries should be sufficiently far from the interested region
and the boundary condition should not affect the calculation results.

Then we move to the core of the Jacobi method. We label the site point in the $\phi^I$ direction by the small character $i$ of its index $I$ 
(or its fraktur $\mathfrak{i}$ for $\pi_I$). Also only the relevant indices will be explicitly expressed.
Then the first order derivative can be given by one of the following discretizations at leading order.
\bae{
	\partial_{\phi^I}u|_i\simeq\frac{u_{i+1}-u}{h_i}
	\simeq\frac{u-u_{i-1}}{h_{i-1}}
	\simeq\frac{u_{i+1}-u_{i-1}}{h_i+h_{i-1}}.
}
The second derivatives are, depending on if the direction is the same or not, respectively given by
\bae{
	 \bce{
	 	\dps
		\partial_{\phi^I}^2u|_i\simeq2\frac{u_{i+1}h_{i-1}+u_{i-1}h_i-u(h_i+h_{i-1})}{h_ih_{i-1}(h_i+h_{i-1})}, \\
		\dps
		\partial_{\phi^I}\partial_{\phi^J}u|_{ij}\simeq\frac{u_{i+1,j+1}-u_{i+1,j-1}-u_{i-1,j+1}+u_{i-1,j-1}}{(h_i+h_{i-1})(h_j+h_{j-1})}.
	}
}
Regarding the discretization of the first derivative, one recalls that the physically fixed boundary is the EoI surface and field values should be calculated 
from this surface. Therefore the discretization is chosen so that the lower-energy point is used, that is, $u_{i+1}$ if $D^I_\phi>0$ and otherwise $u_{i-1}$ is selected.
Let us label the points where $D_\phi^I>0$ as $i_+$ and $i_-$ for other points (and similarly for $\pi$). Then the adjoint FP operator is discretized as
\bae{
	&\hspace{-20pt}\calL_\FP^\dagger\cdot u \nonumber \\
	\simeq&\sum_{i_+}D_\phi^I\frac{u_{i_++1}-u}{h_{i_+}}+\sum_{j_-}D_\phi^J\frac{u-u_{j_--1}}{h_{j_--1}}
	+\sum_{\mathfrak{i}_+}D_{\pi I}\frac{u_{\mathfrak{i}_++1}-u}{h_{\mathfrak{i}_+}}
	+\sum_{\mathfrak{j}_-}D_{\pi J}\frac{u-u_{\mathfrak{j}_--1}}{h_{\mathfrak{j}_--1}} \nonumber \\
	&+\sum_I\left(D_{\phi\phi}{}^{II}\frac{u_{i+1}h_{i-1}+u_{i-1}h_i-u(h_i+h_{i-1})}{h_ih_{i-1}(h_i+h_{i-1})}
	+D_{\pi\pi II}\frac{u_{\mathfrak{i}+1}h_{\mathfrak{i}-1}+u_{\mathfrak{i}-1}h_\mathfrak{i}-u(h_\mathfrak{i}+h_{\mathfrak{i}-1})}
	{h_\mathfrak{i}h_{\mathfrak{i}-1}(h_\mathfrak{i}+h_{\mathfrak{i}-1})}\right) \nonumber \\
	&+\sum_{I\neq J}\left(\frac{1}{2}D_{\phi\phi}{}^{IJ}\frac{u_{i+1,j+1}-u_{i+1,j-1}-u_{i-1,j+1}+u_{i-1,j-1}}{(h_i+h_{i-1})(h_j+h_{j-1})}
	+D_{\phi\pi}{}^I{}_J\frac{u_{i+1,\mathfrak{j}+1}-u_{i+1,\mathfrak{j}-1}-u_{i-1,\mathfrak{j}+1}+u_{i-1,\mathfrak{j}-1}}{(h_i+h_{i-1})(h_\mathfrak{j}+h_{\mathfrak{j}-1})}
	\right.\nonumber \\
	&\hspace{20pt}\left.+\frac{1}{2}D_{\pi\pi IJ}
	\frac{u_{\mathfrak{i}+1,\mathfrak{j}+1}-u_{\mathfrak{i}+1,\mathfrak{j}-1}-u_{\mathfrak{i}-1,\mathfrak{j}+1}+u_{\mathfrak{i}-1,\mathfrak{j}-1}}
	{(h_\mathfrak{i}+h_{\mathfrak{i}-1})(h_\mathfrak{j}+h_{\mathfrak{j}-1})}\right) \nonumber \\
	=&\left[-\sum_{i_+}\frac{D_\phi^I}{h_{i_+}}+\sum_{j_-}\frac{D_\phi^J}{h_{j_--1}}
	-\sum_{\mathfrak{i}_+}\frac{D_{\pi I}}{h_{\mathfrak{i}_+}}+\sum_{\mathfrak{j}_-}\frac{D_{\pi J}}{h_{\mathfrak{j}_--1}}
	-\sum_I\left(\frac{D_{\phi\phi}{}^{II}}{h_ih_{i-1}}+\frac{D_{\pi\pi II}}{h_\mathfrak{i}h_{\mathfrak{i}-1}}\right)\right]u \nonumber \\
	&-\left[-\sum_{i_+}D_\phi^I\frac{u_{i_++1}}{h_{i_+}}+\sum_{j_-}D_\phi^J\frac{u_{j_--1}}{h_{j_--1}}
	-\sum_{\mathfrak{i}_+}D_{\pi I}\frac{u_{\mathfrak{i}_++1}}{h_{\mathfrak{i}_+}}
	+\sum_{\mathfrak{j}_-}D_{\pi J}\frac{u_{\mathfrak{j}_--1}}{h_{\mathfrak{j}_--1}} \right. \nonumber \\
	&\hspace{20pt}-\sum_I\left(D_{\phi\phi}{}^{II}\frac{u_{i+1}h_{i-1}+u_{i-1}h_i}{h_ih_{i-1}(h_i+h_{i-1})}
	+D_{\pi\pi II}\frac{u_{\mathfrak{i}+1}h_{\mathfrak{i}-1}+u_{\mathfrak{i}-1}h_\mathfrak{i}}
	{h_\mathfrak{i}h_{\mathfrak{i}-1}(h_\mathfrak{i}+h_{\mathfrak{i}-1})}\right) \nonumber \\
	&\hspace{20pt}-\sum_{I\neq J}\left(\frac{1}{2}D_{\phi\phi}{}^{IJ}\frac{u_{i+1,j+1}-u_{i+1,j-1}-u_{i-1,j+1}+u_{i-1,j-1}}{(h_i+h_{i-1})(h_j+h_{j-1})} \right. \nonumber \\
	&\hspace{40pt}+D_{\phi\pi}{}^I{}_J\frac{u_{i+1,\mathfrak{j}+1}-u_{i+1,\mathfrak{j}-1}-u_{i-1,\mathfrak{j}+1}+u_{i-1,\mathfrak{j}-1}}{(h_i+h_{i-1})(h_\mathfrak{j}+h_{\mathfrak{j}-1})}
	\nonumber \\
	&\hspace{40pt}\left.\left.+\frac{1}{2}D_{\pi\pi IJ}
	\frac{u_{\mathfrak{i}+1,\mathfrak{j}+1}-u_{\mathfrak{i}+1,\mathfrak{j}-1}-u_{\mathfrak{i}-1,\mathfrak{j}+1}+u_{\mathfrak{i}-1,\mathfrak{j}-1}}
	{(h_\mathfrak{i}+h_{\mathfrak{i}-1})(h_\mathfrak{j}+h_{\mathfrak{j}-1})}\right)\right] \nonumber \\
	=:&\,Cu-U.
}







\subsection{Adiabatic-entropic decomposition}\label{sec: Adiabatic-entropic decomposition}



\section{Configuration space formulation}






%\acknowledgments





%\appendix







%\bibliography{}
\end{document}
